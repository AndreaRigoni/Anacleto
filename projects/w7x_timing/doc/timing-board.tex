% Some macros for logic timing diagrams.
%
% Author: ir. Pascal T. Wolkotte and Jochem Rutgers, University of Twente
\documentclass{article}
\usepackage{verbatim,calc,listings,lscape}
\usepackage[margin=1cm]{geometry}
\begin{comment}
:Title: Timing diagram
:Tags: Diagrams, Macro packages

Uses ``timing.sty``, a convenient set of macros for drawing
logic timing diagrams. Written by Pascal Wolkotte and Jochem Rutgers.

The macros are also available as a separate sty file:

- `timing.sty`_

.. _timing.sty: /media/pgftikzexamples/circuits/timing.sty

:Authors: ir. Pascal T. Wolkotte and Jochem Rutgers, University of Twente
\end{comment}

% Start of timing.sty

% Some macros for logic timing diagrams.
%
% Author: ir. Pascal T. Wolkotte and Jochem Rutgers, University of Twente
% Version: 0.1
% Date: 2007/10/11
\usepackage{tikz}
\newcounter{wavenum}
\setlength{\unitlength}{1pt}
% advance clock one cycle, not to be called directly
\newcommand*{\clki}{
  \draw (t_cur) -- ++(0,.3) -- ++(.5,0) -- ++(0,-.6) -- ++(.5,0) -- ++(0,.3)
    node[time] (t_cur) {};
}
\newcommand*{\bitvector}[3]{
  \draw[fill=#3] (t_cur) -- ++( .1, .3) -- ++(#2-.2,0) -- ++(.1, -.3)
                         -- ++(-.1,-.3) -- ++(.2-#2,0) -- cycle;
  \path (t_cur) -- node[anchor=mid] {#1} ++(#2,0) node[time] (t_cur) {};
}
% \known{val}{length}
\newcommand*{\known}[2]{
    \bitvector{#1}{#2}{white}
}
% \unknown{length}
\newcommand*{\unknown}[2][]{
    \bitvector{#1}{#2}{black!20}
}
% \bit{1 or 0}{length}
\newcommand*{\bit}[2]{
  \draw (t_cur) -- ++(0,.6*#1-.3) -- ++(#2,0) -- ++(0,.3-.6*#1)
    node[time] (t_cur) {};
}
% \unknownbit{length}
\newcommand*{\unknownbit}[1]{
  \draw[ultra thick,black!20] (t_cur) -- ++(#1,0) node[time] (t_cur) {...};
}
% \nextwave{name}
\newcommand{\nextwave}[1]{
  \path (0,\value{wavenum}) node[left] {#1} node[time] (t_cur) {};
  \addtocounter{wavenum}{-1}
}
% \clk{name}{period}
\newcommand{\clk}[2]{
    \nextwave{#1}
    \FPeval{\res}{(\wavewidth+1)/#2}
    \FPeval{\reshalf}{#2/2}
    \foreach \t in {1,2,...,\res}{
        \bit{\reshalf}{1}
        \bit{\reshalf}{0}
    }
}
% \begin{wave}[clkname]{num_waves}{clock_cycles}
\newenvironment{wave}{
  \begin{tikzpicture}[draw=black, yscale=.7,xscale=.58]
    \tikzstyle{time}=[coordinate]
}{\end{tikzpicture}}
% \begin{wave}[clkname]{num_waves}{clock_cycles}
\newenvironment{wave2}[2][10MHz clk]{
  \begin{wave}
    \setcounter{wavenum}{0} 
		\nextwave{#1}
    \foreach \t in {-1,0,...,23}{
      \draw[dotted] (t_cur) +(0,.5) node[above] {\t} -- ++(0,.4-#2);
      \clki
    }
		\unknownbit{1}{1}
		\foreach \t in {88,89,...,101}{
      \draw[dotted] (t_cur) +(0,.5) node[above] {\t} -- ++(0,.4-3);
      \clki
    }
}{\end{wave}}
%%% End of timing.sty

\begin{document}
\thispagestyle{empty}
\part*{\center RedPitaya - Timing Module}
\section*{Firmware}
\subsection*{Data types}
Definition: \textit{boolean} = 1bit logical, \textit{int} = 4byte integer, \textit{long} = 8byte integer.\\
Using \textit{int} as data type for DELAY and SEQUENCE limits the maximal duration to little more than 3.5 minutes. Considering quasi continuous pulses, \textit{long} is used.
Since some diagnostics require arbitrarily distributed clocks the maximum number of elements the Sequence vector should be maxed out by the hardware constrains but at least 1M samples (8 MB, using DRAM).
\subsection*{private parameters - registers}
\begin{tabular}{llll}
address&name&type&description\\\hline
0x0000&INIT&boolean&Used to deinit() the device and break the loops in armed() and program().\\
0x0008&DELAY&long&Number of tick between the input trigger and the beginning of the first sequence.\\
0x0010&WIDTH&int&Number of ticks the output remains high when raised.\\
0x0014&PERIOD&int&Minimum number of ticks between two raising edges:\\
&&& 1. defines the rate of a pulse train.\\
&&& 2. defines when the gate closes after the last pulse.\\
0x0018&CYCLE&long&Number of ticks between two sequence starts (c.f. PERIOD but for the sequence).\\
0x0020&REPEAT&int&Number of sequence repetitions.\\
0x0024&COUNT&int&Number of pulses in a sequence or pulse train.\\
0x0028&SEQUENCE&long[]&Increasing number of tick counts since the sequence start to the raising edge of a pulse.\\
\end{tabular}

\subsubsection*{Startup states}
INIT is cleared, all parameters are set to 0/null, i.e. the whole register is cleared. However, it is satisfactory to ensure the clearing of the first 30 bytes. All outputs should be on low. Unused digital pins should be configured as input.

\subsubsection*{Parameter contrains/defaults}
These constrains should be checked against by the driver before programming.\\
\begin{tabular}{llll}
name  &constrain&default&default*\\\hline
DELAY &$\geq0$&$0$&$60,000,000$\\
WIDTH &$\geq1$&$\rm{WIDTH}/2$&$5$\\
PERIOD&$>\rm{WIDTH}$&$10$&$10$\\
CYCLE &$\geq\rm{COUNT}*\rm{PERIOD}$&$\rm{COUNT}*\rm{PERIOD}$&$0$\\
&$\geq\rm{SEQUENCE[end]}+\rm{PERIOD}$&$\rm{SEQUENCE[end]}+\rm{PERIOD}$&-\\
REPEAT&$\geq0$&$1$&$0$\\
COUNT &$\geq0$&$1$&$0$\\
 &$\geq0,\quad \leq \rm{len}(\rm{SEQUENCE})$&$\rm{len}(\rm{SEQUENCE})$&-\\
\end{tabular}

*) default value if nothing or only DELAY is provided, only valid for makeClock.

\subsection*{Network packages}
The body of makeSequence() and makeClock() could be copied to the register directly as byte stream.\\
\begin{tabular}{l=l:l}
address&name&size&description\\\hline
0x0000&magic&3bytes&Magic bytes, e.g. $0x573758=$'W7X'\\
0x0003&method&1byte&Method identifier, i.e. deinit: 0x00, init1: 0x01, init2: 0x02\\
0x0004&length&4bytes&Length of argument body, i.e. deinit: 0, init2: 40, init1: 40 + 8 count\\
0x0008&delay&8bytes&1nd argument\\
0x0010&width&8bytes&2nd argument\\
0x0014&period&8bytes&3rd argument\\
0x0018&cycle&8bytes&4th argument\\
0x0020&repeat&4bytes&5th argument\\
0x0024&count&4bytes&7th argument\\
0x0028&sequence[0]&8bytes&1st element of sequence argument (only for init1)\\
...
\end{tabular}

The remote programming is currently available via mdsip with provided tdi functions.

\subsection*{Network interrupts (public methods)}
\subsubsection*{arm (method = 0x00)}
INIT is set. This will cause the program to react on trig events.
\subsubsection*{disarm (method = 0x00)}
INIT is cleared. This will interrupt the loops in armed() and program() if running. The device is idling until the next init.
\subsubsection*{makeSequence (method = 0x01)}
The parameters DELAY, WIDTH, PERIOD, CYCLE, REPEAT, COUNT, and SEQUENCE are transmitted. INIT is set. The device is armed().
\subsubsection*{makeClock (method = 0x02)}
The parameters DELAY, WIDTH, PERIOD, CYCLE, REPEAT, and COUNT are transmitted. SEQUENCE is generated from PERIOD and COUNT. INIT is set. The device is armed().
\subsection*{private methods}
\subsubsection*{armed}
The procedure armed() calls getTrigger() until it returns TRUE, in which case armed() executes program(). The device will remain armed until a deinit interrupt.
\subsubsection*{program}
The method program() generates the trigger and gate signal. It is called by armed() when once trigger has occurred.
\subsubsection*{advance}
The method advance(T,G) synchronizes the code. If INIT is set the output channels are set to T and G. Otherwise, they are both cleared. It returns TRUE if INIT was not set.
\subsection*{abstract methods}
\subsubsection*{getTrigger}
The method getTrigger() checks the TRIGGER register and returns TRUE if the trigger level has been high.
\subsubsection*{setOutput}
The method setOutput(T,G) sets the state of trigger\_out and gate\_out to T and G, respectively.
\subsubsection*{sync}
The method sync() synchronizes the code to the with the external clock (raising/falling edge).

\subsection*{Timing}
The board cycle is synchronized to an external clock of $10$Mhz.
The most important aspect is that the jitter is minimized. Fixed delays of a few ticks ($<10$ maybe) are acceptable. The most important part is that the timestamps of the TTLs can be calculate with little uncertainty based on the external clock.

\subsection*{Trigger schema}
At W7X there will be a trigger $T_0=-60$s before the experiment starts at $T_1=0$s. This trigger will engage the initialization of all diagnostics as well as the timing module. Since the timing module should support output signals even seconds before $T_1$ it is required to trigger the timing module with $T_0$. The triggered module should enter the delay phase which should be set to $60$s. During the delay phase the module should accept an reconfiguration (makeClock, makeSequence) without loosing track of the tick count with respect to $T_0$. The updated delay and timing configuration should be used by the device on-the-fly. If the device is not configured before the end or delay, it should return to armed state (i.e. if count=0 or repeat=0).

\clearpage\thispagestyle{empty}
\subsection*{Program}
$\rm{TIME\_WIDTH} \geq 40$
\begin{lstlisting}
architecture Behavioral of w7x_timing is
    -- summary of valid states                             SG0PWA
    constant IDLE           : std_logic_vector(0 to 5) := "000000";
    constant ARMED          : std_logic_vector(0 to 5) := "000001";
    constant WAITING_DELAY  : std_logic_vector(0 to 5) := "000010";
    constant WAITING_SAMPLE : std_logic_vector(0 to 5) := "010110";
    constant WAITING_LOW    : std_logic_vector(0 to 5) := "010010";
    constant WAITING_HIGH   : std_logic_vector(0 to 5) := "110010";
    constant WAITING_REPEAT : std_logic_vector(0 to 5) := "000110";
    constant ERROR          : std_logic_vector(0 to 5) := "000111";
		-- -- signals
    signal   state          : std_logic_vector(0 to 5) := IDLE;
    -- measure number of samples in sequence, i.e. len(times)
    signal reset        : std_logic := '1'; -- start_cycle   =0, do_waiting_sample ++
    signal sample_count : integer := 0; -- start_cycle   =0, do_waiting_sample ++
    signal sample_total : integer := 0;
    -- measure number of repetitions
    signal repeat_count : integer := 0; -- start_program =0, start_waiting_repeat ++
    signal repeat_total : integer := 0;
    -- measure high and low of signal
    signal period_ticks : unsigned(31 downto 0) := (others => '0');
    signal high_total   : unsigned(31 downto 0) := (others => '0');
    signal period_total : unsigned(31 downto 0) := (others => '0');
    -- sequence counter
    signal cycle_ticks  : unsigned(TIME_WIDTH-1 downto 0) := (others => '0');
    signal delay_total  : unsigned(TIME_WIDTH-1 downto 0) := (others => '0');
    signal cycle_total  : unsigned(TIME_WIDTH-1 downto 0) := (others => '0');
    -- contains the next sample of interrest defined by index_out of the previous cycle
    signal curr_sample  : unsigned(TIME_WIDTH-1 downto 0) := (others => '0');
begin
    -- set input
    delay_total  <= unsigned(delay (TIME_WIDTH-1 downto 0));
    high_total   <= unsigned(width);
    period_total <= unsigned(period);
    cycle_total  <= unsigned(cycle (TIME_WIDTH-1 downto 0));
    repeat_total <= to_integer(unsigned(repeat));
    sample_total <= to_integer(unsigned(count));
    curr_sample  <= unsigned(sample(TIME_WIDTH-1 downto 0));
    state_out    <= state;

  clock_gen:  process(clk_in, init_in, trig_in,
                      delay_total,
                      period_ticks, high_total, period_total,
                      cycle_ticks, cycle_total,
                      repeat_total,
                      sample_total, sample_count, curr_sample) is
  
    procedure inc_cycle is
    begin
      cycle_ticks <= cycle_ticks  + 1;
    end inc_cycle;
    
    procedure inc_period is
    begin
      period_ticks <= period_ticks + 1;
      inc_cycle;
    end inc_period;

    procedure start_sample is
    -- resets period_ticks (1)
    begin
      state <= WAITING_HIGH;
      period_ticks <= (0=> '1', others => '0');
    end start_sample;

    procedure do_waiting_sample is
    -- increments sample_count
    begin
      if cycle_ticks = curr_sample then
        sample_count <= sample_count + 1;
        start_sample;
        inc_cycle;
      elsif cycle_ticks > curr_sample then
        state <= ERROR;
      else
        state <= WAITING_SAMPLE;
        inc_cycle;
      end if;
    end do_waiting_sample;

    procedure start_cycle is
    -- resets sample_count (1:0)
    -- resets cycle_ticks  (1)
    begin
      if curr_sample = 0 then -- short cut if first sample is at 0
        sample_count <= 1;
        start_sample;
      else
        sample_count <= 0;
        period_ticks <= (others => '0');
        state <= WAITING_SAMPLE;
      end if;
      cycle_ticks <= (0=> '1', others => '0');
    end start_cycle;
    
    procedure do_waiting_repeat is
    begin
      if cycle_ticks = cycle_total then -- short cut if cycle_total just fits sequence
        start_cycle;
      elsif cycle_ticks > cycle_total then
        state <= ERROR;
      else
        state <= WAITING_REPEAT;
        inc_cycle;
      end if;
    end do_waiting_repeat;

    procedure start_armed is
    -- resets everything
    begin
      cycle_ticks  <= (others => '0');
      period_ticks <= (others => '0');
      sample_count <= 0;
      repeat_count <= 0;
      state <= ARMED;
    end start_armed;
		
    procedure start_waiting_repeat is
    -- increments repeat_count
    begin
      if repeat_count < repeat_total then
        repeat_count <= repeat_count + 1;
        do_waiting_repeat;
      else
        start_armed;
      end if;
    end start_waiting_repeat;

    procedure start_waiting_sample is
    begin
      if sample_count < sample_total then
        do_waiting_sample;       
      else
        start_waiting_repeat;
      end if;
    end start_waiting_sample;     

    procedure do_waiting_low is
    begin
      if period_ticks = period_total then
        start_waiting_sample;
      elsif period_ticks > period_total then
        state <= ERROR;
      else
        state <= WAITING_LOW;
        inc_period;
      end if;
    end do_waiting_low;

    procedure do_waiting_high is
    begin
      if period_ticks = high_total then
        state <= WAITING_LOW;
        inc_period;
      elsif period_ticks > high_total then
        state <= ERROR;
      else
        state <= WAITING_HIGH;
        inc_period;
      end if;
    end do_waiting_high;

    procedure do_waiting_delay is
    begin
      if cycle_ticks = delay_total then
        if repeat_total > 0 and sample_total > 0 then
          start_cycle;
        else
          start_armed;
        end if;
      elsif cycle_ticks > delay_total then
        state <= ERROR;
      else
        state <= WAITING_DELAY;
        inc_cycle;
      end if;
    end do_waiting_delay;

    procedure start_program is
    -- resets repeat_count (1)
    -- resets cycle_ticks (_:1)
    begin
      repeat_count <= 1;
      if delay_total = 0 then -- short cut if no delay
        start_cycle;
      else
        state <= WAITING_DELAY;
        cycle_ticks <= (0=> '1', others => '0');
       end if;
    end start_program;

  begin  -- main program

    if init_in = '0' then
      state <= IDLE;
    elsif rising_edge(clk_in) then
      case state is
        when ARMED => 
          if trig_in = '1' then
            start_program;
          end if;
        when WAITING_DELAY =>
          do_waiting_delay;
        when WAITING_SAMPLE =>
          do_waiting_sample;
        when WAITING_HIGH =>
          do_waiting_high;
        when WAITING_LOW =>
          do_waiting_low;
        when WAITING_REPEAT =>
          do_waiting_repeat;
        when IDLE =>
          start_armed;
        when others =>
          state <= ERROR;
      end case;
    end if; -- rising_edge(clk)
    if sample_count<sample_total then
      index_out <= std_logic_vector(to_unsigned(sample_count,32));
    else
      index_out <= (others => '0');
    end if;

  end process clock_gen;
end Behavioral;

\end{lstlisting}
\clearpage\thispagestyle{empty}
\begin{landscape}
\part*{\center Timing}
\section*{Program}
A program contains the list of strictly monotonic increasing tick counts relative to the trigger input for all pulses that will be generated when a trigger arrives. It is the lowest level of control. A program can be compiled as a initial $\mathrm{DELAY}$ and $\mathrm{REPEAT}$ identical, subsequent cycles.

\begin{wave}
 \nextwave{trigger\_in} \bit{0}{1}\bit{1}{1}\bit{0}{38}
\textbf{ \nextwave{program}\unknown{1}\known{program}{36}\unknown{3}
}
 \nextwave{delay}\unknown{1} \known{delay}{5}\unknown{34}
 \nextwave{N cycles}\unknown{6}\known{1st cycle}{10}\known{2nd cycle}{10}\unknownbit{1}\known{N-th cycle}{10}\unknown{3}
 \nextwave{sequence}\unknown{6}\known{1st sequence}{5}\unknown{5}\known{2nd sequence}{5}\unknown{5}\unknownbit{1}\known{N-th sequence}{5}\unknown{8}
\end{wave}

\section*{Cycle}
During every cycle a digital signal is generated. A gate signal will be high during the sequence and low during idle time. The duration of one cycle is defined by $\mathrm{CYCLE}$ and controls the repetition rate.
\subsection*{Sequence ($\mathrm{WIDTH}=2,\mathrm{PERIOD}=5,\mathrm{SEQUENCE}=\left[3,8,14,21,\cdots,92,96\right]$), init}
The digital signal is generated as a sequence of pulses of a given $\mathrm{WIDTH}$.
The timing of the pulses is defined by $\mathrm{SEQUENCE}$ that contains tick counts relative to the beginning of the cycle. The low-high transition of the gate is defined by the beginning of the cycle. The high-low transition is defined by the last element of $\mathrm{SEQUENCE}$ plus $\mathrm{PERIOD}$.


\begin{wave2}{3}
 \nextwave{signal}\bit{0}{4} \bit{1}{2} \bit{0}{3} \bit{1}{2} \bit{0}{4}\bit{1}{2}\bit{0}{5}\bit{1}{2}\bit{0}{1}\unknownbit{1}\bit{0}{4}\bit{1}{2}\bit{0}{2}\bit{1}{2}\bit{0}{4}
 \nextwave{gate}\bit{0}{1} \bit{1}{38}\bit{0}{1}
\end{wave2}

\subsection*{Pulse train ($\mathrm{WIDTH}=3,\mathrm{PERIOD}=10,\mathrm{COUNT}=10$), init2}
In pulse train mode the parameter $\mathrm{PERIOD}$ and $\mathrm{COUNT}$ are used generate a sequence of equidistant pulses.\hfill
$\mathrm{SEQUENCE} = [0:\mathrm{COUNT}-1]*\mathrm{PERIOD}$


\begin{wave2}{3}
 \nextwave{signal} \bit{0}{1}\bit{1}{3} \bit{0}{7} \bit{1}{3} \bit{0}{7}\bit{1}{3}\bit{0}{1}\unknownbit{1}\bit{0}{2}\bit{1}{3}\bit{0}{9}
 \nextwave{gate} \bit{0}{1}\bit{1}{37}\bit{0}{2}
\end{wave2}
\end{landscape}
\end{document}
